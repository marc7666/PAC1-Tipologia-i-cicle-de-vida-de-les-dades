\documentclass[a4paper,12pt]{report}
\usepackage[catalan]{varioref}
\usepackage{setspace}
\input{sections/packages}
\input{sections/frontPage}
\onehalfspacing

\begin{document}
\thispagestyle{empty}
	\begin{titlepage}
		\maketitle
		\thispagestyle{empty}
	\end{titlepage}
	\cleardoublepage
	\newpage

\thispagestyle{empty}
\tableofcontents
\thispagestyle{empty}
\newpage
\pagenumbering{arabic}
%\thispagestyle{empty}
\section*{Exercici 1}
\addcontentsline{toc}{section}{Exercici 1}
\subsection*{Pregunta 1}
\addcontentsline{toc}{subsection}{Pregunta 1}
Per aplicar la tècnica de normalització caldria, en primer lloc, seleccionar aquelles variables numèriques a les quals volem aplicar la tècnica i després ajustar-les a una escala més reduïda, per tal de, fer-les més comprensibles. Típicament, les escales són entre -1.0 i 1.0 o entre 0.0 i 1.0.\\
Per aplicar la tècnica de discretització cal, com en el cas anterior, seleccionar aquelles variables numèriques a les quals volem aplicar la tècnica. Seguidament, cal substituir els valors numèrics per una sèrie d'etiquetes les quals poden ser conceptuals o intervals. Aquesta tècnica permet consolidar un possible criteri d'agrupació de dades. Per exemple, un \textit{dataset} amb els resultats dels balanços d'empreses. Es podria substituir els resultats dels balanços per 'positiu', 'negatiu' i 'neutre' i, posteriorment, agrupar pel resultat del balanç.\\
Aquestes tècniques són molt importants per als models de predicció, ja que permeten que les dades siguin més senzilles d'interpretar i això comporta que el rendiment del model sigui millor i que aquest tingui la capacitat de trobar patrons més fàcilment, és a dir, el model es fa més senzill.
\subsection*{Pregunta 2}
\addcontentsline{toc}{subsection}{Pregunta 2}
Mentre un enginyer de dades treballa en el desenvolupament, construcció, prova i manteniment d'arquitectures, un científic de dades és una persona que es dedica a obtenir informació rellevant sobre les dades, a partir de preguntes que s'ha plantejat, i a transmetre aquesta informació de manera senzilla. A més, tot i que ambdós perfils tenen coneixements de programació, l'enginyer de dades no aprofundeix tant en coneixements d'estadística i matemàtiques com el científic de dades.\\
\href{https://www.linkedin.com/jobs/search/?currentJobId=4167044480&geoId=104738515&keywords=data%20scientist&origin=JOB_SEARCH_PAGE_SEARCH_BUTTON&refresh=true}{\underline{Oferta de \textit{data scientist}}}\\
\href{https://www.linkedin.com/jobs/search/?currentJobId=4172639382&geoId=104738515&keywords=data%20engineer&origin=JOB_SEARCH_PAGE_SEARCH_BUTTON&refresh=true}{\underline{Oferta de \textit{data engineer}}}

\subsection*{Pregunta 3}
\addcontentsline{toc}{subsection}{Pregunta 3}

\subsection*{Pregunta 4}
\addcontentsline{toc}{subsection}{Pregunta 4}

\subsection*{Pregunta 5}
\addcontentsline{toc}{subsection}{Pregunta 5}

\subsection*{Pregunta 6}
\addcontentsline{toc}{subsection}{Pregunta 6}

\end{document}
